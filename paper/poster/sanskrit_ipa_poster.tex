%%%%%%%%%%%%%%%%%%%%%%%%%%%%%%%%%%%%%%%%%
% Jacobs Landscape Poster
% LaTeX Template
% Version 1.1 (14/06/14)
%
% Created by:
% Computational Physics and Biophysics Group, Jacobs University
% https://teamwork.jacobs-university.de:8443/confluence/display/CoPandBiG/LaTeX+Poster
% 
% Further modified by:
% Nathaniel Johnston (nathaniel@njohnston.ca)
%
% This template has been downloaded from:
% http://www.LaTeXTemplates.com
%
% License:
% CC BY-NC-SA 3.0 (http://creativecommons.org/licenses/by-nc-sa/3.0/)
%
%%%%%%%%%%%%%%%%%%%%%%%%%%%%%%%%%%%%%%%%%

%----------------------------------------------------------------------------------------
%	PACKAGES AND OTHER DOCUMENT CONFIGURATIONS
%----------------------------------------------------------------------------------------

\documentclass[final]{beamer}

\usepackage[scale=1.1,size=a0]{beamerposter} % Use the beamerposter package for laying out the poster



\usetheme{confposter} % Use the confposter theme supplied with this template

\usepackage{tikz}
\usepackage{xltxtra}
\usepackage{calc}
\usepackage{color}
\usepackage{fontspec}

\usetikzlibrary{positioning,calc}

\newlength{\mylength}

\usepackage{multirow, makecell}
\usepackage{tablefootnote}
\usepackage{tabularx}
\usepackage{tipa}
\let\ipa\textipa
\usepackage{vowel,amssymb}
\usepackage{soul}
\usepackage[ruled]{algorithm2e}

\setmainfont{FreeSerif}
\usefonttheme{serif}

\usepackage{polyglossia}
\setmainlanguage{english}
\setotherlanguages{sanskrit} %% or other languages
\newfontfamily\devanagarifont[Script=Devanagari]{Annapurna SIL}
\newcommand{\sansk}[1]{\begin{sanskrit}#1\end{sanskrit}}  % Dummy command; does not do anything



\settowidth\rotheadsize{Minimal}
\newcommand{\rot}[1]{\rotatebox{90}{#1}}
\newcommand{\multirot}[1]{\multirow{2}{*}{\rot{#1}}}

\setbeamercolor{block title}{fg=dblue\iffalse ngreen\fi,bg=white} % Colors of the block titles
\setbeamercolor{block body}{fg=black,bg=white} % Colors of the body of blocks
\setbeamercolor{block alerted title}{fg=white,bg=dblue!70} % Colors of the highlighted block titles
\setbeamercolor{block alerted body}{fg=black,bg=dblue!10} % Colors of the body of highlighted blocks
% Many more colors are available for use in beamerthemeconfposter.sty

%-----------------------------------------------------------
% Define the column widths and overall poster size
% To set effective sepwid, onecolwid and twocolwid values, first choose how many columns you want and how much separation you want between columns
% In this template, the separation width chosen is 0.024 of the paper width and a 4-column layout
% onecolwid should therefore be (1-(# of columns+1)*sepwid)/# of columns e.g. (1-(4+1)*0.024)/4 = 0.22
% Set twocolwid to be (2*onecolwid)+sepwid = 0.464
% Set threecolwid to be (3*onecolwid)+2*sepwid = 0.708

\newlength{\sepwid}
\newlength{\onecolwid}
\newlength{\twocolwid}
\newlength{\threecolwid}
%\setlength{\paperwidth}{142cm} % A0 width: 46.8in
%\setlength{\paperheight}{86cm} % A0 height: 33.1in
\setlength{\sepwid}{0.024\paperwidth} % Separation width (white space) between columns
\setlength{\onecolwid}{0.22\paperwidth} % Width of one column
\setlength{\twocolwid}{0.464\paperwidth} % Width of two columns
\setlength{\threecolwid}{0.708\paperwidth} % Width of three columns
\setlength{\topmargin}{-0.9in} % Reduce the top margin size
%-----------------------------------------------------------

\usepackage{graphicx}  % Required for including images

\usepackage{booktabs} % Top and bottom rules for tables

%----------------------------------------------------------------------------------------
%	TITLE SECTION 
%----------------------------------------------------------------------------------------

\title{A rule-based system for the transcription of Sanskrit from Devanagari to IPA} % Poster title
\author{Aalok Sathe} % Author(s)

\institute{
	University of Richmond
	} % Institution(s)

%----------------------------------------------------------------------------------------
%	BEGIN DOCUMENT
%

\begin{document}

\addtobeamertemplate{block end}{}{\vspace*{2ex}} % White space under blocks
\addtobeamertemplate{block alerted end}{}{\vspace*{2ex}} % White space under highlighted (alert) blocks
%\addtobeamertemplate{frametitle}{}{\vspace{-7em}}

%	UNIVERSITY ICON IN HEADER
\addtobeamertemplate{headline}{} 
{\begin{tikzpicture}[remember picture, overlay]
	\node [anchor=north west, inner sep=1.5cm]  at ($(current page.north west) + (1,-1)$)
	{\includegraphics[scale=.15]{UR-Icon}};
	\end{tikzpicture}} % Logo next to the title
\addtobeamertemplate{headline}{}

\setlength{\belowcaptionskip}{1ex} % White space under figures
\setlength\belowdisplayshortskip{2ex} 
% White space under equations

\begin{frame}[t] % The whole poster is enclosed in one beamer frame
%\vskip-2.9em

%   ALL COLUMNS GO HERE
%%%%%%%%%%%%%%%%%%%%%%%%%%%%%%%%%%%%%%%%%%%%%%%%%%%%%%%%%%%%%%%%%%%%%%%%%%%%%%%%%%%%%%%%%%%%%%%%
%%%%%%%%%%%%%%%%%%%%%%%%%%%%%%%%%%%%%%%%%%%%%%%%%%%%%%%%%%%%%%%%%%%%%%%%%%%%%%%%%%%%%%%%%%%%%%%%
%%%%%%%%%%%%%%%%%%%%%%%%%%%%%%%%%%%%%%%%%%%%%%%%%%%%%%%%%%%%%%%%%%%%%%%%%%%%%%%%%%%%%%%%%%%%%%%%
%%%%%%%%%%%%%%%%%%%%%%%%%%%%%%%%%%%%%%%%%%%%%%%%%%%%%%%%%%%%%%%%%%%%%%%%%%%%%%%%%%%%%%%%%%%%%%%%

        \begin{columns}[t]  % to split into three columns


            \begin{column}{\sepwid}\end{column} % Empty spacer column

            \begin{column}{\twocolwid} % The first column, two columns wide

                %----------------------------------------------------------------------------------------
                %	ABSTRACT
                %----------------------------------------------------------------------------------------


                \setbeamercolor{block alerted body}{fg=black,bg=blue!5}
                \begin{alertblock}{Abstract}
                
                    %\sffamily
                    We propose a system for the transcription of Sanskrit text written using the Devanagari orthography,
                    into the International Phonetic Alphabet, and supplement it with free and open-source software.
                    We make use of existing literature on closest known pronunciations of sounds as well as prosodic
                    and metric rules of syllabification using the Weerasinghe-Wasala-Gamage (WWG) algorithm for Sinhala,
                    adapted to Sanskrit. We further incorporate suprasegmental sound changes along with the assignment of
                    syllable-weight-determined stress.

                \end{alertblock}

                %----------------------------------------------------------------------------------------

	
                %	SANSKRIT PHONOLOGY

                \setbeamercolor{block alerted title}{fg=black,bg=blue!7}
                \setbeamercolor{block alerted body}{fg=black,bg=blue!7}
                %\setbeamerfont{block alerted body}{family=\fontspec{FreeSerif}}
                %\setbeamerfont{block alerted title}{family=\fontspec{FreeSerif}}
                
                \begin{block}{Phonological considerations}
                
                    \begin{itemize}
                    
                        \item {\bfseries Shorthand for nasalization}
                            
                            Nasals may be one of six types: five, derived from
                            the conventional places of articulation (velar, palatal,
                            retroflex, dental, and labial), and the sixth, simply a
                            nasalized articulation of any vowel.
                            Conventionally, a nasal consonant is only explicitly
                            written when a phrase ends, or if the upcoming character is a vowel.
                            In case the nasal sound is not explicitly shown, an {\it anusvara}
                            is shown on the character preceding it, and the actual sound
                            corresponding to it is inferred from the forthcoming sound at the time of reading.
                        
                            \sansk{संस्कृत} $\rightarrow$ ([\textbf{sə̃̃s}.kɹ̩.t̪ə]);\quad
                            \sansk{अंक} $\rightarrow$ ([\textbf{əŋ}.kə])
                        
                        \item {\bfseries Default schwa}
                        
                            A consonant character in Devanagari Sanskrit, unless explicitly
                            marked {\it halant} (i.e., a schwa-less ``partial'' sound marked
                            using the diacritic \sansk{्}), has an implied schwa. For instance,
                            \sansk{ग} may be transcribed as [ɡə], while to yield [ɡ], we would
                            need to mark a lack of schwa as \sansk{ग्}.
                        
                        \item {\bfseries Syllabification}
                        
                    \end{itemize}
                    
                \end{block} % end PHONOLOGICAL considerations
                
                    %   TABLE 2: CONSONANTS
                    
	                \begin{table}[ht]
                        \centering
                        
                        \begin{center}
                            \begin{tabular}{|l|c|c|c|c|c|c|c|}
                                
                                \hline
                                & \textbf{Vl. plosive }&\textbf{Vl. aspirated }&\textbf{Vd. plosive }
                                &\textbf{Vd. aspirated }&\textbf{Nasal}&
                                \textbf{Approximant}&\multicolumn{1}{c|}{\textbf{Fricative}}\\
                                & \textbf{}&\textbf{ plosive}&\textbf{ }&\textbf{ plosive}&\textbf{}&&
                                \\
                                
                                \hline
                                Glottal&&&&&&&ह \quad  [ɦə]$^{**}$\\	\hline
                                Velar&क \quad[kə]	&ख \quad [kʰə]&	ग \quad [gə]&	घ \quad [gʰə]& ङ \quad [ŋə]&&\\\hline
                                Palatal& च \quad [t͡ɕə]&	छ \quad  [t͡ɕʰə]& ज \quad  [d͡ʑə]&	झ \quad  [d͡ʑʱə]& ञ \quad  [ɲə]&य \quad  [jə]&
                                \multirow{2}{*}{श \quad  [ɕə]}\\\cline{1-7}
                                \multirow{2}{*}{Alveolar}&&&&&&र \quad  [ɹə]&\\\cline{8-8}
                                &&&&&&ल  \quad [lə]$^*$&स \quad  [sə]\\\hline
                                Retroflex& ट \quad  [ʈə]&	ठ \quad  [ʈʰə]& 	ड \quad  [ɖə]& ढ \quad  [ɖʰə]&	ण \quad  [ɳə]&ळ \quad  [ɭə]$^*$&ष \quad  [ʂə]\\\hline
                                Dental& त \quad  [t̪ə]& 	थ \quad  [t̪ʰə]&	द \quad  [d̪ə]&		ध \quad  [d̪ʰə]& न \quad  [nə]&\multirow{2}{*}{व \quad  [ʋə]}&\\\cline{1-6}\cline{8-8}
                                Labial & प \quad  [pə]& 	फ \quad  [pʰə]& ब \quad  [bə]& 	भ \quad  [bʱə]& म \quad  [mə]&&\\\hline
                                
                                %\multicolumn{6}{c}{ }\\\hline
                                
                                %\multicolumn{6}{|c|}{\textbf{Other sounds}}\\\hline\hline
                                %Approximants& & && &\\\hline Fricatives&& & 	& 	&\\\hline&
                                %&\multicolumn{4}{c|}{ }	%'क्ष' \quad  'kʂə', 	'ज्ञ' \quad  'd͡ʒɲə', 	'त्र' \quad  't̪ɹə'}
                                %\\\hline
                            \end{tabular}
                            \vskip.5em
                            \caption{\ Sanskrit speech sounds in Devanagari: consonants and
                            non-vowel sounds. Merged cells indicate shared place of articulation.
                                %Orientation of table is so as to align it with conventional Sanskrit speech sounds arrangement. 
                                $^*$Lateral approximants. $^{**}$Voiced fricative.}
                            \label{consonants-table}
                        \end{center}
                        %\label{cons}
                    \end{table}
                   
	
                	
            \end{column}        % end first column (two cols wide)

            \begin{column}{\sepwid}\end{column} % Empty spacer column


            \begin{column}{\onecolwid}      % begin second column (only one col wide)
	

                %	EXAMPLE STIMULI

                \setbeamercolor{block alerted body}{fg=black,} % Change the alert block body colors
                    %   TABLE 1: VOWELS AND SYLLABIC SOUNDS
                    
                    \begin{table}[ht]
	                    \centering
		
	                    \begin{center}
		                    \begin{tabular}{|c|c|c||c|c|c|}
			                    \hline
			                    \textbf{Base}&\multirow{1}{*}{\textbf{Diacritic}}&
			                    \multirow{1}{*}{\textbf{IPA}}&\textbf{Base} &\multirow{1}{*}{\textbf{Diacritic}}&
			                    \multirow{1}{*}{\textbf{IPA}} \\
			                    \hline
			                    \sansk{अ}&&ə& \sansk{आ}& \sansk{ा}&ɑː\\\hline
			                    \sansk{इ}& \sansk{ि}&i&	\sansk{ई}& \sansk{ी}&iː\\\hline
			                    \sansk{उ}&\sansk{ ु}&u&
			                    \sansk{ऊ}& \sansk{ ू}&uː\\\hline
			                    \sansk{ऋ}&\sansk{ ृ}&ɹ̩&\sansk{ॠ} &\sansk{ ॄ}&ɹ̩ː\\\hline
			                    \sansk{ऌ}&\sansk{	ॢ}&l̩&\sansk{ॡ} &\sansk{ॣ}&l̩ː\\\hline
			                    \sansk{ए}&\sansk{ े}&eː& \sansk{ऐ} & \sansk{	ै}&ɑːi\\\hline
			                    \sansk{ओ}& 	\sansk{ो }&oː&	\sansk{	अाै}&\sansk{ाै}&  ɑːu\\\hline
			                    \sansk{अं}& \sansk{	ं }&əm&	\sansk{अः}&	\sansk{ः} & əh\\\hline
			                    \sansk{ॐ} &&oːm&&&%\multicolumn{3}{|c|}{ }
			                    \\\hline
			                    
		                    \end{tabular}
		                    \vskip.5em
		                    \caption{\ Sanskrit speech sounds: vowels and syllabic sounds.}
		                    \label{vowels-table}
		                    
	                    \end{center}
                    \end{table} %   end VOWELS/SYLLABIC SOUNDS table

                
                \setbeamercolor{block alerted title}{fg=black,bg=blue!7}
                \setbeamercolor{block alerted body}{fg=black,bg=blue!7}
                \begin{alertblock}{Algorithm}
                    \begin{algorithm}[H]
	                    \label{alg1}
	                    \SetAlgoLined
	                    \KwIn{Sanskrit text to be syllabified}
	                    \texttt{initialize} scope at the beginning of text\;
	                    \While{end of text not reached}{
		                    \texttt{move} to next $V_B\mathbb{C}V_A$, where $\mathbb{C}$ is a consonant cluster\;
		                    \uIf{length of cluster $\mathbb{C} = 1$}{
			                    \texttt{mark} syllable break after $V_B$\;
		                    }\uElseIf{length of cluster $\mathbb{C} = 2$}{\texttt{mark}
		                    syllable break after first $C$ from left\;}\uElseIf{length of cluster $\mathbb{C} = 3$}{
			                    \uIf{third consonant from left $=$ \sansk{र् or य्} {\bf or} first
			                    and second consonants are stops}{\texttt{mark} syllable break after first $C$ from left\;}
			                    \Else{\texttt{mark} syllable break before first $C$ from right\;}
		                    }\Else{
			                    \uIf{first consonant from right $=$ \sansk{र् or य्}}{\texttt{mark}
			                    syllable break before second $C$ from right\;}
			                    \Else{\texttt{mark} syllable break after least sonorous $C$\;}
			                    }
	                    }\KwResult{Syllabified Sanskrit text
	                    %\vskip0.5em
	                    }
	                \NoCaptionOfAlgo
                    \caption{\bfseries WWG Algorithm adapted to Sanskrit}
                    \end{algorithm}
                
                \end{alertblock}

            \end{column}        % End of the second column

            \begin{column}{\sepwid}\end{column} % Empty spacer column

            \begin{column}{\onecolwid} % The third column


                %	FUTURE RESEARCH

                \setbeamercolor{block alerted title}{fg=black,bg=blue!7}
                \setbeamercolor{block alerted body}{fg=black,bg=blue!7}
                
                \begin{alertblock}{Future exploration}

                    content

                \end{alertblock}

                %----------------------------------------------------------------------------------------
                %	REFERENCES
                %----------------------------------------------------------------------------------------

                \begin{block}{References}

                    % \nocite{*} % Insert publications even if they are not cited in the poster
                    \nocite{zieba2002original,weerasinghe2005rule,gdasa2013sanskritSyllabification}
                    \small{\bibliographystyle{plain}
                    \bibliography{sample}}

                \end{block}

                %----------------------------------------------------------------------------------------
                %	ACKNOWLEDGEMENTS
                %----------------------------------------------------------------------------------------

                \setbeamercolor{block title}{fg=red,bg=white} % Change the block title color

                \begin{block}{Acknowledgements}
                      We are grateful for helpful comments by and discussion with Mukund Gokhale,
                        Hema Kshirsagar, Dieter~Gunkel, Shardul~Chiplunkar, and Thomas Bonfiglio.
                \end{block}\vskip.4em

                %----------------------------------------------------------------------------------------
                %	CONTACT INFORMATION
                %----------------------------------------------------------------------------------------

                \setbeamercolor{block alerted title}{fg=black,bg=norange} % Change the alert block title colors
                \setbeamercolor{block alerted body}{fg=black,bg=white} % Change the alert block body colors

                \begin{alertblock}{Contact information}

                    \begin{itemize}
                        \item Web: \href{https://github.com/aalok-sathe/sanskrit_IPA}{https://github.com/aalok-sathe/sanskrit\textunderscore IPA}
                        \item Email: \href{mailto:aalok.sathe@richmond.edu}{aalok.sathe@richmond.edu}
                    \end{itemize}

                \end{alertblock}


                %----------------------------------------------------------------------------------------

            \end{column} % End of the third column

            \begin{column}{\sepwid}\end{column}         % Empty spacer column
            

        \end{columns}       % End of group of three columns
        
        
%   END COLUMNS
%%%%%%%%%%%%%%%%%%%%%%%%%%%%%%%%%%%%%%%%%%%%%%%%%%%%%%%%%%%%%%%%%%%%%%%%%%%%%%%%%%%%%%%%%%%%%%%%
%%%%%%%%%%%%%%%%%%%%%%%%%%%%%%%%%%%%%%%%%%%%%%%%%%%%%%%%%%%%%%%%%%%%%%%%%%%%%%%%%%%%%%%%%%%%%%%%
%%%%%%%%%%%%%%%%%%%%%%%%%%%%%%%%%%%%%%%%%%%%%%%%%%%%%%%%%%%%%%%%%%%%%%%%%%%%%%%%%%%%%%%%%%%%%%%%
%%%%%%%%%%%%%%%%%%%%%%%%%%%%%%%%%%%%%%%%%%%%%%%%%%%%%%%%%%%%%%%%%%%%%%%%%%%%%%%%%%%%%%%%%%%%%%%%

\end{frame} % End of the enclosing frame

\end{document}
